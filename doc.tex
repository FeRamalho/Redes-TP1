\documentclass[12pt]{article}

% Packages used
\usepackage[T1]{fontenc}
\usepackage[utf8]{inputenc}
\usepackage[margin=1in]{geometry}
\usepackage{color}
\usepackage{hyperref}
\usepackage{graphicx}
\title{\LARGE \textbf{\uppercase{Programação modular\\Trabalho prático 2: Poker Texas Hold'em}} }
\date{12 de junho, 2017}
\author{Rafael Rubbioli\\
\and Manoel da Rocha\\
\and Danilo Viana\\ Departamento de Ciência da computação, UFMG}
\begin{document}
	\begin{titlepage}
		\maketitle
	\end{titlepage}
	\section{Introdução}
	O jogo Poker é um dos jogos mais jogados do mundo. Por isso, neste trabalho implementaremos nossa versão do jogo, com algumas alterações. O objetivo disso é utilizarmos de práticas de projetos orientados a objeto e aprofundarmos no conhecimento de orientação a objeto. Além disso, nos familiarizaremos com o projeto usando diagramas UML e padrões de projeto.
	\section{O jogo}
	Poker é um jogo de apostas extremamente famoso em todo o mundo. Devido a isso, existem várias modalidades do jogo com regras variadas e usaremos uma versão simplificada da modalidade Texas Hold'em. 
    Nosso jogo contém um jogador interativo, ou seja, o humano que jogará e três jogadores que são inteligências
    artificiais simples.
	\subsection{Regras}
        Nossa versão do poker foi simplificada, pois acreditamos que o foco do trabalho seja a utilização correta das
        técnicas de padrão de projetos.
        O jogo começa perguntando ao jogador o preço do blind, valor a ser pago por todos para participar da rodada. Na
        nossa versão o blind é obrigatorio a todos, caso não tenha dinheiro, o jogador é eliminado das rodadas futuras.
        Cada jogador recebe duas cartas, a mesa recebe cinco cartas, primeiramente apenas três são abertas.\\
        A partir deste ponto, as jogadas ocorrem até que a quinta carta seja aberta. O jogador humano é perguntado se
        deseja apostar, caso deseje ele digitara o valor que pretende aumentar a aposta, caso contrario digitará 0.\\
        Em caso de aumento na aposta, os outros jogadores avaliarão se vão participar, caso não queiram, os jogadores
        dão \"fold\", saindo do jogo por esta rodada.
        Caso o jogador humano mantenha a aposta, alguma das AI's poderá aumentar a aposta, e o jogo verificara se todos
        desejam continuar na rodada.
        No final é contabilizado o score do jogador, levando em consideração as regras de mãos de poker original, mas
        possuindo uma pequena diferença. As mãos de carta maior e dupla avaliam a soma das cartas ao invés de avaliar
        qual é a maior do jogador.\\
        Após essa avaliação, é dito qual jogador venceu a rodada e é perguntado se o jogador quer jogar outro round.
		As combinações de mão possíveis do poker original são (Da de maior valor para a de menor valor):
		\begin{enumerate}
			\item[1)]Royal Straight Flush - Sequência “real”( ás, rei, dama, valete e 10) do mesmo naipe.
			\item[2)]Straight Flush - Sequência de cartas do mesmo naipe.
			\item[3)]Four of a kind - Quatro cartas iguais.
			\item[4)]Full House - Uma trinca e uma dupla.
			\item[5)]Flush - Cinco cartas do mesmo naipe.
			\item[6)]Straight - Cinco cartas em sequência.
			\item[7)]Three of a Kind - Trinca de cartas iguais e duas diferentes.
			\item[8)]Two pair - Duas duplas de cartas iguais e uma diferente.
			\item[9)]Pair - Uma dupla de cartas iguais e três cartas diferentes.
			\item[10)]High Card - Não se enquadra em nenhuma das acima.
		\end{enumerate}
	\section{Modelagem do Jogo}
		Para a realização deste trabalho fizemos primeiramente a modelagem. Para isso, precisamos fazer algumas perguntas importantes, como: "Quais são as classes importantes?", "Como podemos modularizar nosso código para podermos usar posteriormente?", "Como podemos decompor esse problema em problemas menores?".
		A resposta dessas perguntas nos gerou dois diagramas importantes. 
		\begin{enumerate}
			\item[1] O primeiro deles é o diagrama de atividades. Esse diagrama nos mostra o que deverá acontecer no fluxo do nosso código durante a execução, ou seja, os possíveis caminhos e resultados gerados por ele.
			\item[2] O segundo é o diagrama de classes. Esse diagrama nos mostra como foi decomposto o problema e a relação entre as classes escolhidas.
		\end{enumerate}
            \includegraphics[scale=0.60]{datividades.png}
            \includegraphics[scale=0.60]{dclasses.png}
		Usamos a Unified Modeling Language(UML), pois ela padroniza e tem um propósito geral. Alem disso, ela prioriza a representação dos elementos essênciais.
		\section{Implementação}
		A implementação do código foi feita na linguagem Java e usamos as classes vistas nos diagramas na seção anterior.
		A classe principal é a classe "Table". Essa classe é chamada de uma classe chamada "Poker", a qual é um
        esqueleto para jogos, o fizemos de acordo com o padrão de projeto Template Method. A classe Table coordena as
        regras e o fluxo do jogo.Para isso, ela conta com atributos como "playersLeft", "pot" e usa funções como
        "betRound" que esta dentro de "runGame" para rodar as rodadas e processar as regras do jogo.
		Fizemos, também, uma classe "HandScorer", que calcula o valor de cada mão, utilizando histogramas com os valores
        das cartas e seus naipes, utilizamos isso pra avaliar o conteudo da mão do jogador, também fazemos verificações
        se o histograma possui valores seguidos para detectar sequências. O score varia como foi visto no entendimento das
        regras do jogo.
		Além disso, criamos as classes dos jogadores. Usamos uma classe chamada "Player" que define os jogadores do jogo. Essa classe controla a mão de cada jogador, seu dinheiro e se ele ainda está em jogo ou não. O que ocorre, porém, é que apenas um dos jogadores é jogável e os outros são jogados pela máquina.
        Isso foi implementado por meio do padrão de projeto Adapter. As jogadas de cada player são processadas pela
        função "agreeBet", que retorna valores inteiros que podem significar: manter a aposta, dar fold ou aumentar a
        aposta. As maiores dificuldades de se implementar um jogo com esse tipo de personagens é o uso de inteligências
        artificiais. No nosso jogo, criamos algo extremamente simples, utilizando histogramas de cartas e naipes
        (conforme na implementação do scorer de mão) para influenciar suas decisões, unidos de alguns elementos de
        aleatoriedade.
        A entrada e saída foi unificada numa classe singleton, para que evitassemos erros devido manipulação de diversos
        scanners sendo instanciados ao longo do código, portanto foi decidido que haveria apenas uma instancia nesta
        classe centralizada.
        \subsection{Dificuldades}
        As maiores dificuldades na realização do trabalho foram a implementação das regras do Poker, que não é um jogo tão simples e as inteligências artificiais. Apesar de tudo, esses não são os propósitos desse trabalho, portanto fizemos uma versão simplificada de ambas as coisas. Isso nos permite focar nas partes principais do trabalho.
        \subsection{Testes}
        Não existe uma maneira fácil de testar algorítmos com resultados aleatórios e entradas de usuário, portanto fizemos um volume de testes grande para tentar garantir o funcionamento correto do jogo. A idéia principal foi tentar usar casos em que imaginávamos que o algorítmo poderia dar erros. Os testes foram bem sucedidos, porém isso não nos permite ter certeza do funcionamento correto em todos os casos.
		\section{Conclusão}
		Neste trabalho vimos que, primeiramente, a modelagem e o projeto são passos muito importantes na criação de
        software. Isso foi visível quando depois de terminado o projeto, a implementação foi muito mais simples e o
        trabalho final teve muito mais qualidades. Outra observação é que uso dos princípios de OO é muito útil e torna
        o código mais legível e reutilizável. Por exemplo, se quiséssemos reutilizar a classe "Card" ou "Deck" em
        qualquer outro jogo de cartas, seria possível. Caso tivessemos a necessidade de utilizar outra AI, poderia ser
        simples, uma vez que utilizamos o padrão adapter, e temos como adicionar mais uma clausula a ele, sem ferir o
        código previamente feito. Vemos assim que o uso de padrões de projeto e os padroes UML foram muito úteis para
        padronizar o trabalho, tornando-o assim, mais fácil de atingir o entendimento e facilita sua expansão e extensão
        para futuros projetos. Por fim, vimos que a utilização de inteligências artificiais não é uma tarefa simples,
        porém tomamos como desafio para implementar uma mais simples como um desafio para incrementar nosso jogo.

		\section{Bibliografia}
        Material de aula.\\
		Especificações da linguagem Java, pela oracle.\\
		Java, como programar. Oitava edição (Deitel).\\
        \end{document}
