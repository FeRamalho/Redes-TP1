\documentclass[10pt]{article}

% Packages used
\usepackage[T1]{fontenc}
\usepackage[utf8]{inputenc}
\usepackage[margin=1in]{geometry}
\usepackage{color}
\usepackage{hyperref}
\usepackage{graphicx}
\title{\LARGE \textbf{\uppercase{REDES DE COMPUTADORES\\Trabalho prático 1A}} }
\date{2 de setembro 2017}
\author{Rafael Rubbioli : 2014124838\\
\and Fernanda Ramalho : 2014106368 \\ Departamento de Ciência da computação, UFMG}
\begin{document}
	\maketitle 
	\section{Introdução}
		Este trabalho tem como objetivo a criação de um emulador de enquadramento de dados da camada de enlace. Ele será feito na linguagem C e usará sockets. Contará com sequenciamento e detecção de erros e será full duplex.
	\section{Desenvolvimento}
		\subsection{Enquadramento}
			Enquadramento de dados é uma técnica muito usada em redes, na qual se coloca informações adicionais nos pacotes enviados. Essas informações podem ser sobre destinatários, remetentes, tamanho do pacote ou mecanismos de verificação / correção de erro. 
			\\ Neste trabalho iremos fazer enquadramento por contagem, ou seja, colocaremos no quadro o tamanho do payload. Além disso, usaremos um mecanismo de sentinelas para "avisar" o receptor de que está chegando um quadro. Colocaremos também um campo "reserved" no quadro, mas não o usaremos.
			\\ O método de detecção de erros usado será o Checksum. Neste método usa-se um algorítmo que calcula a soma dos dados do quadro antes de enviálos e logo após de recebe-los. Caso a soma seja igual, não houve erros e tudo pode continuar, mas caso ocorra erros, descartaremos todo o pacote até encontrar novos bytes de sincronização. Esse método não é o mais eficiente de detecção de erros que existe, visto que se houvesse um erro que mantivesse a soma igual, o pacote errado seria considerado correto, mas é bastante simples e este não é o objetivo principal deste trabalho.
			\subsubsection{Formato}
			O quadro será do seguinte formato: 
			\\8 bytes de \textbf{synch} (2x DCC023C2)
			\\2 bytes de \textbf{checksum} (calculado pelo algorítmo)
			\\2 bytes de \textbf{length}  (tamanho do payload)
			\\2 bytes de \textbf{reserved}
			\\length/8 bytes de \textbf{payload} (dados realmente usados)
		\subsection{Entrada e saída}
			O programa deverá receber um arquivo de entrada e saída. O arquivo de entrada será lido (em binário), enquadrado e passado pela camada de enlace para que, do outro lado seja verificado e, caso esteja correto escrito no arquivo de saída (em binário).
			Além disso, o programa deverá receber o IP da rede, a porta (ou porto) e modo que a conexão deverá ser feita (ativa ou passiva). Exatamente nesta ordem :
			
			\textbf{./frame INPUT OUTPUT IP PORT MODE}
		\subsection{Transmissão}
			A transmissão dos dados é feita como já foi descrito anteriormente. Primeiro o arquivo é lido de um em um byte, até que o tamanho máximo permitido no payload seja obtido, e enquadrado. O tamanho dos dados lidos é calculado enquanto os bytes são lidos. 
			\\Depois disso, é feito o checksum e o quadro inteiro é transmitido. A maneira como a transmissão é feita não importa, desde que todos os dados sejam enviados corretamente.
			\\A função que implementa a transmissão termina quando o arquivo inteiro é lido. A conexão é fechada. Assim, para continuar a mandar os dados, é preciso fazer outra conexão.
		\subsection{Recepção}
			A recepção dos dados é feita de maneira que nada é lido até chegarem bytes de sincronização, depois é lido o restante do quadro e feito o checksum. Caso o checksum seja igual ao recebido no quadro, os dados são escritos no arquivo de saída.
			\\O arquivo de saída é aberto em modo "ab", ou seja, permite que escreva no final do arquivo, sem modificar o que já está lá. Em caso de vários quadros, os payloads recebidos serão colocados no final.
			\\A função que implementa o receptor só para de esperar por quadros quando o cliente fecha a conexão. 
		\subsection{Full Duplex}
			O programa rodará da maneira full duplex, ou seja, fará o envio e o recebimento dos dados ao mesmo tempo, independentemente do modo em que esteja rodando (ativo ou passivo). Para isso, usamos threads. 
			\\ A thread é um "desvio" controlado no fluxo de execução onde criamos mais de um fluxo no programa. Neste caso, precisamos criar apenas um fluxo adicional para fazer o recebimento de dados, já que o fluxo principal estava fazendo a transmissão dos dados.
	\section{Terminação}
		O programa é terminado quando o nó remoto fecha a conexão ou quando recebe um sinal de interrupção do teclado (ctrl + c).
	\section{Dificuldades}
		As maiores dificuldades do trabalho foram em trabalhar com os tipos da linguagem C. O problema se encontra nos \textit{casts} e nas converções de tipos.
		\\
		bla bla bla
	\section{Testes}
		Os testes foram feitos rodando o programa em dois terminais diferentes e com um testbench que envia quadros errados. Além disso, foram usados vários tipos de arquivos como .txt, .pdf, .jpg etc
		A corretude do programa foi verificada.
	\section{Conclusão} 
		Funciona.
\end{document}
